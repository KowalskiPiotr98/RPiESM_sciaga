\documentclass[10pt,landscape,a4paper,notitlepage]{article}
\usepackage[utf8]{inputenc}
\usepackage[polish]{babel}
\usepackage{polski}
\usepackage{amsmath}
\usepackage{amsfonts}
\usepackage{multicol}
\usepackage{marvosym}
\usepackage[left=0.5cm,right=0.5cm,top=0.5cm,bottom=0.5cm]{geometry}
\author{Piotr Kowalski}
\title{Ściąga na egzamin z Rachunku Prawdopodobieństwa i Elementów Statystyki Matematycznej}

\begin{document}
    \begin{multicols*}{4}
        [
        \begin{center}
            \MaleMale \FemaleFemale Ściąga na egzamin z Rachunku Prawdopodobieństwa i Elementów Statystyki Matematycznej - \textcopyright Piotr Kowalski, \today , wersja 0.2, MIT License \FemaleFemale \MaleMale
        \end{center}
        ]
        \noindent\textbf{\large Zdarzenie elementarne}\\
        Każdy możliwy wynik eksperymentu losowego nazywamy \textbf{zdarzeniem elementarnym} $\omega$, a zbiór wszystkich możliwych wyników eksperymentu (wszystkich zdarzeń elementarnych) nazywamy \textbf{zbiorem zdarzeń elementarnych} i oznaczamy $\Omega, (\omega\in\Omega)$.
        
        \noindent\textbf{\large Aksjomaty prawdopodobieństwa}\\
        Dla danego zbioru zdarzeń elementarnych $\Omega$ oraz $\sigma$-ciała zdarzeń losowych $\mathcal{F}$, \textbf{prawdopodobieństwem} nazywamy funkcję $P:\mathcal{F}\rightarrow \mathcal{R}$ spełniającą:\\
        1. Dla dowolnego zdarzenia losowego $A\in\mathcal{F}$, $P(A)\geq 0$.\\
        2. $P(\Omega)=1$.\\
        3. Dla dowolnego nieskończonego ciągu zdarzeń losowych $A_1, A_2, \ldots, \forall_{n\in\mathcal{N}} A_n\in\mathcal{F}$, parami rozłącznych, mamy $P\left(\bigcup^\infty_{n=1}\right)=\sum^\infty_{n=1}P(A_n)$.

        \noindent \textbf{Dla dowolnych zdarzeń} $A, B$ mamy\\ $P(A\cup B)=P(A)+P(B)-P(A\cap B)$.

        \noindent \textbf{\large Prawdopodobieństwo warunkowe}\\
        Prawdopodobieństwo $A$ pod warunkiem że zaszło zdarzenie $B$: $P(A|B)=\frac{P(A\cup B)}{P(B)}$.\\
        Jeżeli $P(A_1\cap \ldots \cap A_n)>0$, to $P(A_1\cap \ldots \cap A_n) = P(A_1)\prod_{i=2}^nP(A_i|A_1 \cap\ldots\cap A_{i-1})$.

        \noindent \textbf{\large Prawdopodobieństwo zupełne}\\
        Ciąg zdarzeń nazywamy zupełnym, jeśli:\\
        1. $\bigcup_i A_i = \Omega$,\\
        2. $\forall_{i\neq j}A_i \cap A_j = \emptyset$,\\
        3. $\forall_i P(A_i)>0$.\\
        \textbf{Twierdzenie}\\
        Jeśli zdarzenia tworzą układ zupełny, to dla dowolnego zdarzenia $B$ mamy $P(B)=\sum_iP(B|A_i)P(A_i)$

        \noindent \textbf{\large Reguła Bayesa}\\
        \textbf{Twierdzenie} Niech $A_i$ tworzą układ zupełny. Wtedy dla dowolnego zdarzenia losowego $B, P(B)>0$ i dowolnego $j$ zachodzi $P(A_j|B)=\frac{P(B|A_j)P(A_j)}{\sum_iP(B|A_i)P(A_i)}$

        \noindent \textbf{\large Niezależność zdarzeń}\\
        \textbf{Definicja} Zdarzenia są wzajemnie niezależne gdy $P(A\cap B)=P(A)\cdot P(B)$\\
        Jeżeli zdarzenia $A$ i $B$ są niezależne, to niezależne są również zdarzenia $A$ i $\overline{B}$, $\overline{A}$ i $B$, $\overline{A}$ i $\overline{B}$.\\
        Zdarzenia są wzajemnie niezależne jeśli $P\left(\bigcap_{j=1}^kA_{i_j}\right) = \prod_{j=1}^kP(A_{i_j})$.\\
        Jeśli $A_1\ldots$ są zdarzeniami wzajemnie niezależnymi, to $\left(\bigcup_{i=1}^nA_i\right)=1-\prod_{i=1}^n(1-P(A_i))$

        \noindent\textbf{\large Łączenie prawdopodobieństw}\\
        szeregowe: $P(A_s)=\prod_{i=1}^np_i$\\
        równoległe: $P(A_r)=1-\prod_{i=1}^n(1-p_i)$

        \noindent\textbf{\large Dystrybuanta}\\
        $F(x)=P(X\leq x) = P(\{\omega\in\Omega : X(\omega)\leq x\})$\\
        Własności:\\
        $\lim_{x\rightarrow-\infty}F(x)=0$, $\lim_{x\rightarrow\infty}F(x)=1$\\
        niemalejąca, prawostronnie ciągła\\
        $P(a<X\leq b)=F(b)-F(a)$

        \noindent\textbf{\large Zmienne losowe dyskretne}\\
        $p(a) = P(X=a)$ - funkcja prawdopodobieństwa, własności:\\
        $p(x)\geq 0$, $\sum_{x\in X}p(x)=1$\\
        Dystrybuanta dyskretna zmiennej $X$ o nośniku $\chi$: $F(x)=\sum_{\{x_i\in \chi:x_i\leq x\}}p(x_i)$.\\
        Przykład:
        \[
            F(x)=
            \begin{cases}
                0, x<1\\
                0.4, 1\leq x<2\\
                0.9, 2\leq x
            \end{cases}
        \]
        
        \noindent \textbf{\large Zmienne losowe ciągłe}\\
        $P(X\in B)=\int_{B} f(x)\,\mathrm{d}x$\\
        $F(x)=\int_{-\infty}^xf(t)\,\mathrm{d}t$\\
        $f(x)=F'(x)$\\
        $P(a\leq X\leq b)=F(b)-F(a)$\\
        $P(X = a)=0$ dla dowolnego $a\in R$\\
        Własności funkcji gęstości:\\
        $\forall_{x\in R}f(x)\geq 0$, $\int_{-\infty}^{+\infty}f(x)\,\mathrm{d}x=1$

        \noindent \textbf{\large Doświadczenie Bernoulliego}\\
        Doświadczenie kończące się sukcesem z prawdopodobieństwem p lub porażką z prawdopodobieństwem 1-p.\\
        Ciąg n doświadczeń z prawd. sukcesu p oznaczamy $b(n,p)$.\\
        Prawdopodobieństwo uzyskania ciągu składającego się z k sukcesów, przy założeniu niezależności: $p^k(1-p)^{n-k}$.\\
        Prawdopodobieństwo uzyskania k sukcesów w n niezależnych doświadczeniach z $p\in[0,1]$: $b(k;n,p)=\binom{n}{k}p^k(1-p)^{n-k}$.\\
        \textbf{Poisson} (fr. Ryba)\\
        Dla $n\geq25$ i $\lambda = n\cdot p \leq10$ możemy przybliżyć rozkładem Poi\ss ona: $b(k;n,p)\approx e^{-np}\frac{(np)^k}{k!}$, na przykład:\\
        $\sum_{k=0}^{14}b(k;500,0.02)\approx F(14;500\cdot0.02)$, gdzie $F$ jest dystrybuantą rozkładu Ryby, dostępna w tablicach.
    \end{multicols*}
\end{document}