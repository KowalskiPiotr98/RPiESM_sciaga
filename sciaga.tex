\documentclass[10pt,landscape,a4paper,notitlepage]{article}
\usepackage[utf8]{inputenc}
\usepackage[polish]{babel}
\usepackage{polski}
\usepackage{amsmath}
\usepackage{amsfonts}
\usepackage{multicol}
\usepackage[left=0.5cm,right=0.5cm,top=0.5cm,bottom=0.5cm]{geometry}
\author{Piotr Kowalski}
\title{Ściąga na egzamin z Rachunku Prawdopodobieństwa i Elementów Statystyki Matematycznej}

\begin{document}
    \begin{multicols*}{4}
        [
        \begin{center}
            Ściąga na egzamin z Rachunku Prawdopodobieństwa i Elementów Statystyki Matematycznej - Piotr Kowalski, \today , wersja 0.5.2-BETA
        \end{center}
        ]
        \noindent\textbf{\large Zdarzenie elementarne}\\
        Każdy możliwy wynik eksperymentu losowego nazywamy \textbf{zdarzeniem elementarnym} $\omega$, a zbiór wszystkich możliwych wyników eksperymentu (wszystkich zdarzeń elementarnych) nazywamy \textbf{zbiorem zdarzeń elementarnych} i oznaczamy $\Omega, (\omega\in\Omega)$.
        
        \noindent\textbf{\large Aksjomaty prawdopodobieństwa}\\
        Dla danego zbioru zdarzeń elementarnych $\Omega$ oraz $\sigma$-ciała zdarzeń losowych $\mathcal{F}$, \textbf{prawdopodobieństwem} nazywamy funkcję $P:\mathcal{F}\rightarrow \mathcal{R}$ spełniającą:\\
        1. Dla dowolnego zdarzenia losowego $A\in\mathcal{F}$, $P(A)\geq 0$.\\
        2. $P(\Omega)=1$.\\
        3. Dla dowolnego nieskończonego ciągu zdarzeń losowych $A_1, A_2, \ldots, \forall_{n\in\mathcal{N}} A_n\in\mathcal{F}$, parami rozłącznych, mamy $P\left(\bigcup^\infty_{n=1}A_n\right)=\sum^\infty_{n=1}P(A_n)$.

        \noindent \textbf{Dla dowolnych zdarzeń} $A, B$ mamy\\ $P(A\cup B)=P(A)+P(B)-P(A\cap B)$.

        \noindent \textbf{\large Prawdopodobieństwo warunkowe}\\
        Prawdopodobieństwo $A$ pod warunkiem że zaszło zdarzenie $B$: $P(A|B)=\frac{P(A\cup B)}{P(B)}$.\\
        Jeżeli $P(A_1\cap \ldots \cap A_n)>0$, to $P(A_1\cap \ldots \cap A_n) = P(A_1)\prod_{i=2}^nP(A_i|A_1 \cap\ldots\cap A_{i-1})$.

        \noindent \textbf{\large Prawdopodobieństwo zupełne}\\
        Ciąg zdarzeń nazywamy zupełnym, jeśli:\\
        1. $\bigcup_i A_i = \Omega$,\\
        2. $\forall_{i\neq j}A_i \cap A_j = \emptyset$,\\
        3. $\forall_i P(A_i)>0$.\\
        \textbf{Twierdzenie}\\
        Jeśli zdarzenia tworzą układ zupełny, to dla dowolnego zdarzenia $B$ mamy $P(B)=\sum_iP(B|A_i)P(A_i)$

        \noindent \textbf{\large Reguła Bayesa}\\
        \textbf{Twierdzenie} Niech $A_i$ tworzą układ zupełny. Wtedy dla dowolnego zdarzenia losowego $B, P(B)>0$ i dowolnego $j$ zachodzi $P(A_j|B)=\frac{P(B|A_j)P(A_j)}{\sum_iP(B|A_i)P(A_i)}$

        \noindent \textbf{\large Niezależność zdarzeń}\\
        \textbf{Definicja} Zdarzenia są wzajemnie niezależne gdy $P(A\cap B)=P(A)\cdot P(B)$\\
        Jeżeli zdarzenia $A$ i $B$ są niezależne, to niezależne są również zdarzenia $A$ i $\overline{B}$, $\overline{A}$ i $B$, $\overline{A}$ i $\overline{B}$.\\
        Zdarzenia są wzajemnie niezależne jeśli $P\left(\bigcap_{j=1}^kA_{i_j}\right) = \prod_{j=1}^kP(A_{i_j})$.\\
        Jeśli $A_1\ldots$ są zdarzeniami wzajemnie niezależnymi, to $\left(\bigcup_{i=1}^nA_i\right)=1-\prod_{i=1}^n(1-P(A_i))$

        \noindent\textbf{\large Łączenie prawdopodobieństw}\\
        szeregowe: $P(A_s)=\prod_{i=1}^np_i$\\
        równoległe: $P(A_r)=1-\prod_{i=1}^n(1-p_i)$

        \noindent\textbf{\large Dystrybuanta}\\
        $F(x)=P(X\leq x) = P(\{\omega\in\Omega : X(\omega)\leq x\})$\\
        Własności:\\
        $\lim_{x\rightarrow-\infty}F(x)=0$, $\lim_{x\rightarrow\infty}F(x)=1$\\
        niemalejąca, prawostronnie ciągła.\\
        $P(a<X\leq b)=F(b)-F(a)$

        \noindent\textbf{\large Zmienne losowe dyskretne}\\
        $p(a) = P(X=a)$ - funkcja prawdopodobieństwa, własności:\\
        $p(x)\geq 0$, $\sum_{x\in X}p(x)=1$\\
        Dystrybuanta dyskretna zmiennej $X$ o nośniku $\chi$: $F(x)=\sum_{\{x_i\in \chi:x_i\leq x\}}p(x_i)$.\\
        Przykład:
        \[
            F(x)=
            \begin{cases}
                0, x<1\\
                0.4, 1\leq x<2\\
                1, x\geq 2
            \end{cases}
        \]
        
        \noindent \textbf{\large Zmienne losowe ciągłe}\\
        $P(X\in B)=\int_{B} f(x)\,\mathrm{d}x$\\
        $F(x)=\int_{-\infty}^xf(t)\,\mathrm{d}t$\\
        $f(x)=F'(x)$\\
        $P(a\leq X\leq b)=F(b)-F(a)$\\
        $P(X = a)=0$ dla dowolnego $a\in R$\\
        Własności funkcji gęstości:\\
        $\forall_{x\in R}f(x)\geq 0$, $\int_{-\infty}^{+\infty}f(x)\,\mathrm{d}x=1$
        

        \noindent\textbf{\large Funkcje zmiennych losowych}\\
        \textbf{Dyskretne:}\\ $P(Y=y_i)=\sum_{\{x_j\in\chi:g(x_j)=y_i\}}P(X=x_j)$\\
        \textbf{Ciągłe:} gęstość liniowej funkcji zm.los.\\ $\forall_{a\neq 0}b\in R f_{aX+b}(y)=\frac{1}{|a|}f_X\left(\frac{y-b}{a}\right)$\\
        gęstość kwadratu zm.los.\\
        \[
            f_{X^2}(y)=
            \begin{cases}
                0, x\leq 0\\
                \frac{1}{2\sqrt{y}}\left[f_X(\sqrt{y})+f_X(-\sqrt{y})\right], x>0
            \end{cases}
        \]
        Jeśli $g$ jest ściśle monotoniczna i różniczkowalna, to $Y=g(X)$: $f_Y(y)=f_X(g^{-1}(y))\cdot\frac{1}{|g'(g^{-1}(y))|}$

        \noindent\textbf{\large Wartość oczekiwana}\\
        Dyskretne: $E[X]=\sum_{x_i\in X}x_i\cdot P(X=x_i)$\\
        Ciągłe: $E[X]=\int_{-\infty}^{+\infty}xf(x)\,\mathrm{d}x$\\
        Dla typu mieszanego o $F(x)=pF_d(x)+(1-p)F_c(x),\,E[X]=pE[X_d]+(1-p)E[X_c]$.\\
        Dla funkcji zmiennej losowej $Y=g(X)$:
        {
            \tiny
            \[
                E[g(X)]=
                \begin{cases}
                    \sum_ig(x_i)P(X=x_i),\,\text{jeśli } X \text{ jest zm.los. dyskretną}\\
                    \int_{-\infty}^{+\infty}g(x)f_X(x)\mathrm{d}x,\,\text{jeśli } X \text{ jest zm.los. ciągłą}
                \end{cases}
            \]
        }\\
        Jeśli istnieje wartość oczekiwana $E[X]$, to $E[aX+b]=aE[X]+b$

        \noindent\textbf{\large Momenty}\\
        Momentem rzędu $n$-tego względem $c\in R$ zmiennej losowej $X$ nazywamy: $E[(X-c)^n]$\\
        Momenty zwykłe: $c=0$\\
        Pierwszy moment: $E[X]$\\
        Jeśli istnieje $n$-ty moment zwykły, to istnieją wszystkie momenty rzędu mniejszego od $n$\\
        Momenty centralne: $c=E[X]$\\
        Wariancja: $V(X)=\sigma_x^2=\sigma^2=E[(X-\mu)^2]=E[X^2]-(E[X])^2$, gdzie $\mu=E[X]$\\
        $V(aX+b)=V(aX)=a^2V(X)$\\
        Odchylenie standardowe: $\sigma=\sqrt{V(X)}$
        Zmienna $X$ jest standaryzowana jeśli $E[X]=0$ i $V(X)=1$\\
        Standaryzacja: $X^*=\frac{X-\mu}{\sigma}$\\
        Współczynnik skośności: $\gamma_1=E\left[\left(\frac{X-\mu}{\sigma}\right)\right]=\frac{E[(X-\mu)^3]}{(E[(X-\mu)^2])^{3/2}}$\\
        Kurtoza: $\gamma_2=\frac{\mu_4}{\sigma^4}-3$

        \noindent\textbf{\large Kwantyle, kwartyle}\\
        Kwantyl: $\forall_{p\in(0,1)} x_p=Q(p)=F^{-1}(p)=\inf \{x\in R:p\leq F(x)\}$\\
        Mediana: kwantyl $x_{0.5}$ rzędu $0.5$.\\
        Dolny kwartyl $Q_1$ - kwantyl rzędu $0.25$, górny kwartyl $Q_3$ - kwantyl rzędu $0.75$\\
        Rozkład międzykwartylowy $IQR=Q_3-Q_1$

        \noindent \textbf{\large Doświadczenie Bernoulliego}\\
        Doświadczenie kończące się sukcesem z prawdopodobieństwem p lub porażką z prawdopodobieństwem 1-p.\\
        Ciąg n doświadczeń z prawd. sukcesu p oznaczamy $b(n,p)$.\\
        Prawdopodobieństwo uzyskania ciągu składającego się z k sukcesów, przy założeniu niezależności: $p^k(1-p)^{n-k}$.\\
        Prawdopodobieństwo uzyskania k sukcesów w n niezależnych doświadczeniach z $p\in[0,1]$: $b(k;n,p)=\binom{n}{k}p^k(1-p)^{n-k}$.\\
        \textbf{Poisson} (fr. Ryba)\\
        Dla $n\geq25$ i $\lambda = n\cdot p \leq10$ możemy przybliżyć rozkładem Poi\ss ona: $b(k;n,p)\approx e^{-np}\frac{(np)^k}{k!}$, na przykład:\\
        $\sum_{k=0}^{14}b(k;500,0.02)\approx F(14;500\cdot0.02)$, gdzie $F$ jest dystrybuantą rozkładu Ryby, dostępna w tablicach.

        \noindent\textbf{\large Rozkłady zmiennych dyskretnych}\\
        \textbf{Bernoulli}\\
        1. Eksperyment składa się z $n$ mniejszych doświadczeń, zwanych próbami,
        gdzie $n$ jest ustalone i znane przed doświadczeniem.
        2. Każda próba kończy się sukcesem lub porażką.
        3. Próby są niezależne.\\
        $X$ ma rozkład Bernoulliego jeśli
        \[
            p(x)=P(X=x)=
            \begin{cases}
                1-p,\,x=0\\
                p,\,x=1
            \end{cases}
        \]\\
        $E[X]=p,\,V(X)=p(1-p)$\\
        \textbf{Dwumianowy, czyli dalej Bernoulli}\\
        Zmienną losową $X$ równą liczbie sukcesów w $n$ niezależnych doświadczeniach Bernoulliego nazywamy zmienną o rozkładzie dwumianowym z param. $(n,p)$, oznaczamy $X\sim b(n,p)$.\\
        $P(X=k)=b(k;n,p)=\binom{n}{k}p^k(1-p)^{n-k}$\\
        $E[X]=np,\,V(X)=np(1-p)$\\
        \textbf{Hipergeometryczny}\\
        1. Losujemy ze zbioru $N$ elementów.
        2. Każdemu z $N$ obiektów można przypisać sukces lub porażka (mamy $M$ sukcesów).
        3. Wybieramy $n$ obiektów bez zwracania tak, że wybór każdego $n$ elementowego podzbioru ma to samo prawdopodobieństwo.\\
        Liczba sukcesów $X$ ma rozkład hipergeometryczny $X\sim HG(n,M,N)$
        $P(X=x)=h(x;n,M,N)=\frac{\binom{M}{x}\binom{N-M}{n-x}}{\binom{N}{n}},\,E[X]=\frac{nM}{N},\,V(X)=\left(\frac{N-n}{N-1}\right)\frac{nM}{N}\left(1-\frac{M}{N}\right)$\\
        \textbf{Ujemny dwumianowy (Pascala)}\\
        1. Niezależne próby Bernoulliego. 2. Eksperyment kończy się w chwili uzyskania $r$-tego sukcesu, $r$ ustalone.\\
        $X$ - Liczba porażek do uzyskania $r$-tego sukcesu ma rozkład ujemny dwumianowy
        $X\sim NB(r,p),\,nb(x;r,p)=\binom{x+r-1}{r-1}p^r(1-p)^x,\,E[X]=\frac{r(1-p)}{p},\,V(X)\frac{r(1-p)}{p^2}$\\
        \textbf{Geometryczny}\\
        $X$ równa liczbie porażek w ciągu niezależnych doświadczeń Bernoulliego, do uzyskania pierwszego sukcesu, ma rozkład geometryczny.\\
        Dla $Y=X+1,\,E[Y]=\frac{1}{p},\,V(Y)=\frac{1-p}{p^2}$\\
        \textbf{Poisson}, The Sequel\\
        1. Istnieje $\lambda>0$, że w dowolnie krótkim przedziale czasowym $\Delta t$ prawdop. zaobserwowania dokładnie jednego zdarzenia wynosi $\lambda\cdot\Delta t+\circ(\Delta t)$.
        2.Prawdop. zaobserwowania w $\Delta t$ więcej niż jednego zdarzenia wynosi $\circ(\Delta t)$.
        3. Liczba zaobserwowanych zd.los. w $\Delta t$ jest niezależna od liczby wcześniej zaobserwowanych zdarzeń.\\
        Jeśli 1-3 spełnione, to $P_k(t)$, że liczba zdarzeń zaobserwowanych do chwili $t$ jest równa $k$ wynosi $P_k(t)=e^{-\lambda t}\frac{(\lambda t)^k}{k!},\,E[X]=\lambda=V(X)$

        \noindent\textbf{\large Rozkłady zmiennych ciągłych}\\
        \textbf{Jednostajny}\\
        $X$ ma rozkład jednostajny na $[a,b],\,X\sim U(a,b)$ jeśli $f(x)=\frac{1}{b-a}$ dla $x\in[a,b]$ i $0$ dla pozostałych.
        Dla $x\in[a,b),\,F(x)=\frac{x-a}{b-a}$.\\
        $E[X]=\frac{a+b}{2},\,V(X)=\frac{(b-a)^2}{12}$\\
        \textbf{Normalny}\\
        $X\sim N(\mu, \sigma^2),\,f(x)=\frac{1}{\sigma\sqrt{2\pi}}\exp\left[-\frac{(x-\mu)^2}{2\sigma^2}\right]$\\
        $F(x)=P(X\leq x)=\Phi(\frac{x-\mu}{\sigma})$, gdzie $\Phi$ patrz poniżej,
        $E[X]=\mu,\,V(X)=\sigma^2$\\
        Jeśli $X$ ma normalny, to $Y=aX+b\sim N(a\mu+b,a^2\sigma^2)$\\
        \textbf{Standardowy normalny}\\
        $U\sim N(0,1)$, gęstość: $\phi(x)=\frac{1}{\sqrt{2\pi}}e^{-\frac{1}{2}x^2}$, $\Phi(x)=\frac{1}{\sqrt{2\pi}}\int_{-\infty}^xe^{-t^2/2}\mathrm{d}t$.\\
        \textbf{Wykładniczy}\\
        Własność braku pamięci: $P(X>s+t|X>t)=P(X>s)$\\
        $X\sim \exp(\lambda)$, $f(x)=\lambda e^{-\lambda x},\,x\geq 0$ i $0$ gdy $x<0$.
        $F(x)=1-e^{-\lambda x},\,x\geq 0$.\\
        \textbf{Gamma}\\
        Gamma Eulera:\\
        $\Gamma(\alpha)=\int_0^{\infty}x^{\alpha-1}e^{-x}\mathrm{d}x,\,\alpha>0$\\
        Własności:\\
        1. $\Gamma(\alpha+1)=\alpha\cdot\Gamma(\alpha)$\\
        2. $\forall_{n\in N}\Gamma(n)=(n-1)!$\\
        3. $\forall_{x\in(0,1)}\Gamma(x)\Gamma(1-x)=\frac{\pi}{\sin(x\pi)}$\\
        $X\sim \gamma(\alpha,\beta),\,f(x)=\frac{\beta^{\alpha}}{\Gamma(\alpha)}x^{\alpha-1}e^{-\beta x}$ dla $x\geq 0$ i $0$ w p.p.\\
        $aX\sim\gamma(\alpha,\frac{\beta}{a}),\,E[X]=\frac{\alpha}{\beta},\,V(X)=\frac{\alpha}{\beta^2}$

        \noindent\textbf{\large Zmienne losowe dwuwymiarowe}\\
        dystrybuanta: $F(x,y)=P(X\leq x,Y\leq y)$\\
        Własności: 
        1. $\lim_{x\rightarrow-\infty}F(x,y)=0,\,\lim_{y\rightarrow-\infty}F(x,y)=0,\,\lim_{x,y\rightarrow\infty}F(x,y)=1$.
        2. Niemalejąca ze względu na każdą ze zmiennych.
        3. Prawostronnie ciągła ze względu na każdą ze zmiennych.
        4. $F(x_1,y_1)+F(x_2,y_2)-F(x_1,y_2)-F(x_2,y_1)\geq 0$\\
        \textbf{Dyskretne}\\
        funkcja prawdop. $p(x,y)=P(X=x,Y=y)$\\
        Własności:
        1. $p(x,y)\geq 0$
        2. $\sum_x\sum_yp(x,y)=1$\\
        Rozkłady brzegowe X: $p_X(x)=P(X=x)=\sum_{y\in Y}p(x,y)$, y analogicznie\\
        Zmienne $X$ i $Y$ są niezależne, gdy $p(x,y)=p_X(x)\cdot p_Y(y)$\\
        \textbf{Ciągłe}\\
        $f$ gęstość, jeśli: 
        $P\{(X,Y)\in A\}=\iint_Af(x,y)\mathrm{d}x\mathrm{y}$\\
        $f(x,y)=\int_{-\infty}^x\int_{-\infty}^yf(u,v)\mathrm{d}u\mathrm{d}v$\\
        $f(x,y)\geq 0,\,\iint_{R^2}f(x,y)\mathrm{d}x\mathrm{d}y=1$\\
        Gęstości brzegowe X: $f_X(x)=\int_{-\infty}^{\infty}f(x,y)\mathrm{d}y$, y analogicznie\\
        Zmienne $X$ i $Y$ są niezależne, gdy $F(x,y)=F_X(x)\cdot F_Y(y)$, co jest równoważne $f(x,y)=f_X(x)\cdot f_Y(y)$\\
        \textbf{Funkcje zmiennych losowych}\\
        Dyskretne:\\
        Jeśli $X_1,\ldots,X_n$ - niezależnie zmienne o jednakowych rozkładach Bernoulliego, to $S_n=\sum_{i=1}^nX_i\sim b(n,p)$\\
        Jeśli $X_i\sim b(n_i,p)$, to $\sum_{i=1}^n\sim b(\sum_{i=1}^nn_i,p)$.\\
        Jeśli $X_i\sim P(\lambda_i)$, to $\sim_{i=1}^nX_i\sim P(\sum_{i=1}^n\lambda_i)$\\
        Ciągłe:\\
        Niech $Z=g(X,Y)$. Dystrybuanta $F_Z(z)=P(Z\leq z)=\iint_{\{(x,y):g(x,y)\leq z\}}f(x,y)\mathrm{d}x\mathrm{d}y$\\
        Gęstość sumy $Z=X+Y:\,f_Z(z)=\int_{-\infty}^{\infty}f(x,z-x)\mathrm{d}x$, 
        gdy $X$ i $Y$ niezależne, to $f_{X+Y}(z)=\int_{-\infty}^{\infty}f_X(x)f_Y(z-x)\mathrm{d}x$\\
        Jeśli $X_i\sim N(\mu_i,\sigma_i^2)$ (niezależne), to $\sum_{i=1}^n\sim N(\sum_{i=1}^n,\sum_{i=1}^n\sigma_i^2)$\\
        Jeśli $X_i\sim \gamma(\alpha_i,\beta)$, to $\sum_{i=1}^nX_i\sim \gamma(\sum_i\alpha_i,\beta)$\\
        $E[h(X,Y)]=\sum_x\sum_yh(x,y)p(x,y)$ dla dyskretnych, $=\int_{-\infty}^{\infty}\int_{-\infty}^{\infty}h(x,y)f(x,y)dxdy$\\
        $E[aX+bY]=aE[X]+bE[Y],\,V(aX+bY)=a^2V(X)+b^2V(Y)+2ab\mathrm{Cov}(X,Y)$\\
        $\mathrm{Cov}(X,Y)=E(X,Y)-E(X)E(Y)$\\
        Współczynnik korelacji: $\rho_{X,Y}=\frac{\mathrm{Cov}(X,Y)}{\sigma_x\cdot\sigma_y}$\\
        \textbf{Regresja liniowa}\\
        Jeśli $(X,Y)$ dwuwymiarowa zmienna losowa, dla której istnieją $\sigma_X^2,\,\sigma_Y^2$ oraz Cov, to $E\{[Y-(aX+b)]^2\}$ osiąga minimum gdy $a=\rho_{XY}\frac{\sigma_Y}{\sigma_X}$ oraz $b=m_Y-am_X$
        tzn. funkcja regresji zmiennej $Y$ przy $X=x$ jest równa $\frac{y-m_Y}{\sigma_Y}=\rho_{XY}\frac{x-m_X}{\sigma_X}$

        \noindent\textbf{\large Przybliżenie rozkładów}\\
        \textbf{Przybliżenie normalne}\\
        $P(S_n\leq x)\approx \Phi(\frac{S_n-n\mu}{\sigma\sqrt{n}})$, gdzie $E[x_i]=\mu,\,V(X_i)=\sigma^2,\,n\geq 50$\\
        \textbf{Rozkładu dwumianowego rozkładem normalnym}\\
        Jeśli $S_n\sim b(n,p),\,n\geq 25,\, n\geq 5,\, n(1-p)\geq 5$, to $P(S_n\leq x)\approx\Phi\left(\frac{x+0.5-np}{\sqrt{np(1-p)}}\right)$\\
        \textbf{Rozkładu Poissona}\\
        $F(k;\lambda)\approx \Phi\left(\frac{k+0.5-\lambda}{\sqrt{\lambda}}\right),\,\lambda>10$

        \noindent\textbf{\large Statystyka}\\
        Rozstęp $R=\max\{x_1,\ldots,x_n\}-\min\{x_1,\ldots,x_n\}$\\
        Średnia:\\
        dla danych niepogrupowanych (naprawdę musisz to sprawdzać?)\\
        dla danych pogrupowanych: $\overline{x}=\frac{1}{n}\sum_{j=1}^kn_j\overline{x_j}$\\
        Wariancja: niepogrupowane: $s^2=\frac{1}{n-1}\sum_{i=1}^n(x_i-\overline{x}_n)^2$\\
        pogrupowane: $s^2=\frac{1}{n-1}\sum_{j=1}^kn_j(\overline{x_j}-\overline{x_n})^2$\\
    \end{multicols*}
\end{document}